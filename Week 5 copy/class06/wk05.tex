% Options for packages loaded elsewhere
\PassOptionsToPackage{unicode}{hyperref}
\PassOptionsToPackage{hyphens}{url}
%
\documentclass[
]{article}
\title{Week 5: R functions}
\author{Pham Vo}
\date{2/7/2022}

\usepackage{amsmath,amssymb}
\usepackage{lmodern}
\usepackage{iftex}
\ifPDFTeX
  \usepackage[T1]{fontenc}
  \usepackage[utf8]{inputenc}
  \usepackage{textcomp} % provide euro and other symbols
\else % if luatex or xetex
  \usepackage{unicode-math}
  \defaultfontfeatures{Scale=MatchLowercase}
  \defaultfontfeatures[\rmfamily]{Ligatures=TeX,Scale=1}
\fi
% Use upquote if available, for straight quotes in verbatim environments
\IfFileExists{upquote.sty}{\usepackage{upquote}}{}
\IfFileExists{microtype.sty}{% use microtype if available
  \usepackage[]{microtype}
  \UseMicrotypeSet[protrusion]{basicmath} % disable protrusion for tt fonts
}{}
\makeatletter
\@ifundefined{KOMAClassName}{% if non-KOMA class
  \IfFileExists{parskip.sty}{%
    \usepackage{parskip}
  }{% else
    \setlength{\parindent}{0pt}
    \setlength{\parskip}{6pt plus 2pt minus 1pt}}
}{% if KOMA class
  \KOMAoptions{parskip=half}}
\makeatother
\usepackage{xcolor}
\IfFileExists{xurl.sty}{\usepackage{xurl}}{} % add URL line breaks if available
\IfFileExists{bookmark.sty}{\usepackage{bookmark}}{\usepackage{hyperref}}
\hypersetup{
  pdftitle={Week 5: R functions},
  pdfauthor={Pham Vo},
  hidelinks,
  pdfcreator={LaTeX via pandoc}}
\urlstyle{same} % disable monospaced font for URLs
\usepackage[margin=1in]{geometry}
\usepackage{graphicx}
\makeatletter
\def\maxwidth{\ifdim\Gin@nat@width>\linewidth\linewidth\else\Gin@nat@width\fi}
\def\maxheight{\ifdim\Gin@nat@height>\textheight\textheight\else\Gin@nat@height\fi}
\makeatother
% Scale images if necessary, so that they will not overflow the page
% margins by default, and it is still possible to overwrite the defaults
% using explicit options in \includegraphics[width, height, ...]{}
\setkeys{Gin}{width=\maxwidth,height=\maxheight,keepaspectratio}
% Set default figure placement to htbp
\makeatletter
\def\fps@figure{htbp}
\makeatother
\setlength{\emergencystretch}{3em} % prevent overfull lines
\providecommand{\tightlist}{%
  \setlength{\itemsep}{0pt}\setlength{\parskip}{0pt}}
\setcounter{secnumdepth}{-\maxdimen} % remove section numbering
\ifLuaTeX
  \usepackage{selnolig}  % disable illegal ligatures
\fi

\begin{document}
\maketitle

\hypertarget{example-input-vectors-to-start-with}{%
\section{Example input vectors to start
with}\label{example-input-vectors-to-start-with}}

student1 \textless- c(100, 100, 100, 100, 100, 100, 100, 90) student2
\textless- c(100, NA, 90, 90, 90, 90, 97, 80) student3 \textless- c(90,
NA, NA, NA, NA, NA, NA, NA)

\#Q1. Write a grade() function to determine an overall grade from a
vector

\hypertarget{to-find-the-position-of-the-smallest-value-i.e.-min-value-in-our-vector}{%
\section{To find the position of the smallest value (i.e.~min) value in
our
vector}\label{to-find-the-position-of-the-smallest-value-i.e.-min-value-in-our-vector}}

student1{[}which.min(student1){]}

mean(student1{[}-which.min(student1){]})

\#So lets map/change the NA values to zero. Use the ``is.na()'' function

is.na(student2)

x \textless- student2 x

x{[}is.na(x){]}b \textless- 0 x mean(x)

\#Combine our working snippets to find the average for figure 3

grade \textless- function(x) \{ x{[}is.na(x){]} \textless- 0
mean(x{[}-which.min(x){]}) \}

grade(student3)

\#Your final function should be adquately explained with code comments
and be able to work on an example class gradebook such as this one in
CSV format:

url \textless- ``\url{https://tinyurl.com/gradeinput}'' gradebook
\textless- read.csv(url, row.names=1) gradebook

apply(gradebook, 1, grade)

\#Q2. Who is the top scoring student overall in the grade book?

result \textless- apply(gradebook, 1, grade) sort(result, decreasing =
TRUE)

which.max(result)

\#Q3. Which homework was toughest on students?

hw.ave \textless- apply(gradebook, 2, mean, na.rm=TRUE)
which.min(hw.ave)

hw.med \textless- apply(gradebook, 2, median, na.rm=TRUE)
which.min(hw.med)

\#There is a different rest when using mean and median. Good idea to
plot the data and see.

boxplot(gradebook)

\#Q4. From your analysis of the gradebook, which homework was most
predictive of overall score (i.e.~highest correlation with average grade
score)?

result gradebook{[}is.na(gradebook){]} \textless- 0 cor(result,
gradebook\$hw5, )

apply(gradebook, 2, cor, x=result)

\#Q5. Make sure you save your Rmarkdown document and can click the
``Knit'' button to generate a PDF foramt report without errors. Finally,
submit your PDF to gradescope. {[}1pt{]}

\end{document}
